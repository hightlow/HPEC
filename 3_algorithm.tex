\section{Bisection and Twisted Algorithm} \label{sec:algorithm}
For an arbitrary matrix $A\in \mathbb{R}^{m \times n} (m>n)$,
The SVD of an arbitrary matrix $A\in \mathbb{R}^{m \times n} (m>n)$ is the form 
\[A = U \Sigma V^T,\] where $U$ and $V$ is a $m \times m$ and $n \times n$ orthogonal matrix whose columns are singular vectors, and $\Sigma$ is a $m\times n$ diagonal matrix whose diagonal elements are singular values.
A typical SVD algorithm can be broken down into two steps \cite{65SIAM}.
The first step is to reduce the initial matrix to bidiagonal matrix by Householder transform.
The second step is to diagonalize the bidiagonal matrix.
Householder transform above has been optimized in paper \cite{LiuHouseholder}.
In this paper, we focus only on the second step that reduces a bidiagonal matrix to a diagonal matrix, for which the ``bisection and twisted'' algorithm is designed.
 
The bisection and twisted algorithm is divided into two phases:
(1) obtain the singular values of the bidiagonal matrix by bisection approach; and
(2) obtain the corresponding left and right singular vectors of every singular value by twisted factorization.
Next we describe these two phases in the following subsections.

\subsection{Bisection Algorithm}\label{subsec:bisection}
Suppose $B$ is an upper bidiagonal $n \times n$ matrix with elements $b_{i,j}$.
The matrix $T = B^T B - \mu^2 I_n$ can be decomposed as
\begin{equation}
T = B^T B - \mu^2 I_n = L D L^T ,
\label{eq:T}
\end{equation}
where $\mu$ is a variable, $D=diag(d_1,d_2,\cdots,d_n)$ and $L$ is lower bidiagonal matrix with the form 
%, whose diagonal elements are $1$s and one-order subdiagonal elements are $l_{i}, (i=1,\cdots,n-1)$. The form of $L$ is
\begin{equation}
 L =  \left( \begin{array}{ccccc}
     1&      &       &        &  \\
 l_{1}& 1    &       & 0      &  \\
      & l_{2}& \ddots&        &  \\
      & 0    & \ddots& \ddots &  \\
      &      &       & l_{n-1}& 1
\end{array} \right) .
\label{eq:l}
\end{equation}
By substituting $L$ and $D$ into Eq. (\ref{eq:T}), we can get the following equations
\begin{equation}
\left \{ \begin{aligned}
b_{1,1}^2 - \mu^2 &= d_1\\
b_{k-1,k-1} b_{k-1,k} &= d_{k-1} l_{k-1}\\
b_{k-1,k}^2 + b_{k,k}^2 - \mu^2 &= l_{k-1}^2 d_{k-1} + d_k
\end{aligned} \right .
\label{eq:ldl}
\end{equation}
where $k = 2,3,\cdots,n$.
Next, we define an auxiliary values $t_{k} = t_{k-1} * (b_{k-1,k}^2 / d_{k-1}) - \mu^2$, and then all the equations in Eq. (\ref{eq:ldl}) reduce to
\begin{equation}
\left \{
\begin{aligned}
d_1 = b_{1,1}^2 - \mu^2 \\
d_k = b_{k,k}^2 + t_{k}
\end{aligned}
\right .
\label{eq:negcount}
\end{equation}

We define a function denoted by $NegCount(\mu)$ to count the number of negative elements of $d_i\;(i=1\cdots n)$ in matrix $D$.
Matrices $D$ and $T$ have the same properties on $NegCount$ function, according to the Sylvester's law of inertia \cite{13sylvester}.
Algorithm \ref{alg:negcount} describes how $NegCount$ function works.
$NegCount(\mu)$ is a monotonically non-decreasing function of $\mu$, the $i$th singular value $x_i$ can be solved by the following equations
\begin{equation}
\left \{
\begin{aligned}
NegCount(x_i+\delta) > i \\
NegCount(x_i-\delta) < i ,
\end{aligned}
\right .
\label{eq:negcount}
\end{equation}
which can be calculated by bisection algorithm \cite{95ETNAbisecion}.
%\alglanguage{pseudocode}
\begin{algorithm}[h]
%\small
\caption{NegCount in Bisection Algorithm}
\label{alg:negcount}
\begin{algorithmic}[1]
\Procedure{$\mathbf{NegCount}$}{$n, B, \mu$}
  \State $d=1$, $t=0$, $cnt=0$, $b_{0,1}=0$;
  \For {$k = 1 \to n$}
    \State $t = t * (b_{k-1,k}^2 / d) - \mu^2$;
    \State $d = b_{k,k}^2 + t$;
    \If {$d < 0$}
      \State $cnt++$;
    \EndIf
  \EndFor
  \State \Return $cnt$;
\EndProcedure
\end{algorithmic}
\end{algorithm}


\subsection{Twisted Algorithm}
Suppose $\lambda$ is a singular value of bidiagonal matrix $B \in R^{n \times n}$.
Then the matrix $B^T B - \lambda^2 I$ can be decomposed as
\begin{equation}
B^T B - \lambda^2 I_n = L D_L L^T = U D_U U^T ,
\end{equation}
where $D_L=diag(\alpha_1, \cdots, \alpha_n)$, $D_U = diag(\beta_1, \cdots, \beta_n)$, 
$L$ share the same format as in Eq. (\ref{eq:l}).
$U$ is upper bidiagonal matrix, who is symmetical to $L$ expect the one-order superdiagonal are $u_{i}, (i=1,\cdots,n-1)$. 

Given the $LDL^T$ and $UDU^T$ decomposition, the twisted factorization of the shifted matrix is shown as
\begin{equation}
B^T B - \lambda^2 I = N_k D_k N_k^T ,
\label{eq:twisted}
\end{equation}
where $D_k$ is diagonal matrix, $N_k$ is a twisted combination of the partial lower triangular $L$ and the partial upper triangular $U$. The form of $N_k$ is in \cite{09NLAAtwisted}.
%The subscript \begin{equation}k = \arg \min \gamma,\end{equation} and $\gamma = (\beta_1, \beta_2 - \alpha_2 * l_1, \cdots, \beta_i - \alpha_i * l_{i-1}, \cdots, \beta_{n-1} - \alpha_{n-1} * l_{n-2}, \alpha_n)$.

Thus, the corresponding singular vector of $\lambda$ can be obtained by solving the following matrix equations and then normalizing the solution:
\begin{equation}
N_k^T Z_k = e_k,
\label{eq:unnorm}
\end{equation}
 where each column in matrix $Z_k$ is the non-normalization solution of singular vector corresponding to $\lambda$, and $e_k$ is a $k$th unit vector.

%One advantage of BT algorithm is that it can be applied to cases where singular values are clustered together.
%\textcolor{red}{
Suppose $m$ is the multiplicity of singular values of matrix $B$. 
The algorithm selects the indices of the first $m$ minimum values of $\gamma$. %(May need some more clarification????)} 
Each index $k$ in $m$ has a different twisted factorization, and thus the singular vector can be obtained by solving its own Eq. (\ref{eq:unnorm}). These singular vectors are also orthogonal to each others\cite{09NLAAtwisted}.
%%\alglanguage{pseudocode}
\begin{algorithm}
%\small
\caption{Twisted Factorization}
\label{alg:twisted}
\begin{algorithmic}[1]
\Procedure{$\mathbf{Twisted}$}{$q, n, B, \mu$}
  \LineComment $l$ is the number of different singular values, $l\le n$;
  \For {$i = 1 \to l$}
    \State Computes matrix $S = B - \mu_i^2 I$;
    \State Computes $LDL'$ decomposition $S = LD_LL'$;
    \State Computes $UDU'$ decomposition $S = UD_UU'$;
    \State Computes $\gamma$ based on Eq \ref{eq:gamma};
    \State Find the number $m$ of clustered singular values $\mu$;
    \State Find $m$-th minimum $k = min_j  |\gamma(j)|$;
    \For {each k}
    \State $z_k = 1$, $z_{k-1} = -L_{k-1,k}$, $z_{k+1} = -U_{k,k+1}$;
    \For {$j = k+2 \to n$}
      \State $z_j = -U_{j-1,j}*z_{j-1}$
    \EndFor
    \For {$j = k-2 \to 1$}
      \State $z_j = -L_{j+1,j}*z_{j+1}$
    \EndFor
    \State Scale vector $q = z/||z||_2$;
    \EndFor
  \EndFor
\EndProcedure
\end{algorithmic}
\end{algorithm}

%Algorithm \ref{alg:twisted} is the algorithm to obtain the corresponding singular vectors of given singular values.
%Lines 4-7 are two obtain the parameter $\gamma$.
%Obtain the multiplicity of a singular value and their corresponding minimum values are described in Lines 8-9.
%Lines 10-16 are to calculate the singular vectors and normalize them.
%%We completely parallelize the twisted factorization to speed up the algorithm on GPU architecture.
%The cost for every singular vector transformation is $O(n)$, and the total cost of the transformations is $O(n^2)$.

