\documentclass{llncs}

\usepackage{ifpdf}
\ifpdf
  \usepackage[pdftex]{graphicx}
  \graphicspath{{./pdf/}{./jpeg/}{./eps/}}
  \DeclareGraphicsExtensions{.pdf,.jpeg,.png}
\else
  \usepackage[dvips]{graphicx}
  \graphicspath{{./eps/}}
  \DeclareGraphicsExtensions{.eps}
\fi

\usepackage{color}
\usepackage[cmex10]{amsmath}
\usepackage{amsfonts}
\usepackage[noend]{algpseudocode}
\usepackage{algorithm}
\algnewcommand{\LineComment}[1]{\State \(\triangleright\) #1}
\usepackage[tight,footnotesize]{subfigure}
\begin{document}

\title{Attachment: Bisection and Twisted Bidiagonal SVD on GPU for Big Matrices}
\section{Algorithms}
Algorithm \ref{alg:negcount} is the $NegCount()$ function. It calculates the singular value that are less than $\mu$ of a $n\times n$ matrix $B$.
%\alglanguage{pseudocode}
\begin{algorithm}[h]
%\small
\caption{NegCount in Bisection Algorithm}
\label{alg:negcount}
\begin{algorithmic}[1]
\Procedure{$\mathbf{NegCount}$}{$n, B, \mu$}
  \State $d=1$, $t=0$, $cnt=0$, $b_{0,1}=0$;
  \For {$k = 1 \to n$}
    \State $t = t * (b_{k-1,k}^2 / d) - \mu^2$;
    \State $d = b_{k,k}^2 + t$;
    \If {$d < 0$}
      \State $cnt++$;
    \EndIf
  \EndFor
  \State \Return $cnt$;
\EndProcedure
\end{algorithmic}
\end{algorithm}


Algorithm \ref{alg:bisection} is the Bisection Algorithm. It gets the singular values in the interval of $[l,u)$ of a $n\times n$ matrix $B$. $val$ is the output of the function containing the singular value between $[l,u)$. $n_l$ and $n_u$ are the indeces of singular values.
%\alglanguage{pseudocode}
\begin{algorithm}
%\small
\caption{Bisection Algorithm}
\label{alg:bisection}
\begin{algorithmic}[1]
\Procedure{$\mathbf{Bisection}$}{$val, n, B, l, u, n_l, n_u,\tau$}
  \If {$n_l \geq n_r$ or $l > u$}
    \State No singular values in $[l,u)$;
  \EndIf
  \State Enqueue ($l, u, n_l, n_u$) to Worklist;
  \While {Worklist is not empty}
    \State Dequeue ($a, b, n_a, n_b$) from Worklist;
    \If {$b - a < \tau$}
      \For {$i=n_a+1 \to n_b$}
        \State $val[i] = \min(\max((a+b)/2,a),b)$;
      \EndFor
    \Else
      \State $m = MidPoint(a,b)$;
      \State NegCount($n_m, B, m$);
      \State $n_m = \min(\max(n_m,n_a),n_b)$;
      \If {$n_m > m_a$}
        \State Enqueue ($a, m, n_a, n_m$) to Worklist;
      \EndIf
      \If {$n_m < m_b$}
        \State Enqueue ($m, b, n_m, n_b$) to Worklist;
      \EndIf
    \EndIf
  \EndWhile
\EndProcedure
\end{algorithmic}
\end{algorithm}


\vspace{1in}
Algorithm \ref{alg:twisted} is the algorithm calculates the singular vectors.
%\alglanguage{pseudocode}
\begin{algorithm}
%\small
\caption{Twisted Factorization}
\label{alg:twisted}
\begin{algorithmic}[1]
\Procedure{$\mathbf{Twisted}$}{$q, n, B, \mu$}
  \LineComment $l$ is the number of different singular values, $l\le n$;
  \For {$i = 1 \to l$}
    \State Computes matrix $S = B - \mu_i^2 I$;
    \State Computes $LDL'$ decomposition $S = LD_LL'$;
    \State Computes $UDU'$ decomposition $S = UD_UU'$;
    \State Computes $\gamma$ based on Eq \ref{eq:gamma};
    \State Find the number $m$ of clustered singular values $\mu$;
    \State Find $m$-th minimum $k = min_j  |\gamma(j)|$;
    \For {each k}
    \State $z_k = 1$, $z_{k-1} = -L_{k-1,k}$, $z_{k+1} = -U_{k,k+1}$;
    \For {$j = k+2 \to n$}
      \State $z_j = -U_{j-1,j}*z_{j-1}$
    \EndFor
    \For {$j = k-2 \to 1$}
      \State $z_j = -L_{j+1,j}*z_{j+1}$
    \EndFor
    \State Scale vector $q = z/||z||_2$;
    \EndFor
  \EndFor
\EndProcedure
\end{algorithmic}
\end{algorithm}

\begin{equation}
\label{eq:gamma}
k = \arg \min_{1\le i \le n} \gamma_{i} = \arg \min_{1\le i \le n}
\begin{cases}
\beta_1 & \text{if } i=1 \\
\beta_i - \alpha_i * l_{i-1} & \text{if } 2\le i\le n-1\\
\alpha_n & \text{if } i=n
\end{cases}
\end{equation}
where $\alpha_i$, $\beta_i$, and $l_i$ are mentioned in Section Algorithm.

\vspace{1in}
Algorithm \ref{alg:lengthsub} is the algorithm that separate the whole interval containing all singular values into equal-length subinterval.
\begin{algorithm}
%\small
\caption{Equal Length Subinterval Algorithm}
\label{alg:lengthsub}
\begin{algorithmic}[1]
\Procedure{$\mathbf{Length\_Divide}$}{$n, B, t, \tau$}
  \State Obtain singular value boundary $[l,u)$ of matrix $B$;
  \State Define the thread ID $i$, $i<t$;
  \State Obtain average step $s = (u-l) / t $;
  \State Lower bound $\alpha = l + i * s$;
%  \State Upper bound $\beta = max(l + (i+1) * s, \alpha)$;
  \State $n_{\alpha} = NegCount(\alpha)$;
%  \State $n_{\beta} = NegCount(\beta)$;
%  \State $\alpha = \min(\alpha,\beta)$;
%  \State $\beta = \max(\alpha, \beta)$;
%  \State $n_{\alpha} = \min(n_{\alpha},n_{\beta})$;
%  \State $n_{\beta} = \max(n_{\alpha}, n_{\beta})$;
%  \State Max scan to refine $\alpha, \beta, n_{\alpha}, n_{\beta}$
%  \State Call $Bisection(val, n, B, \alpha, \beta, n_{\alpha}, n_{\beta}, \tau)$
  \State save the division point $\alpha$ and $n_{\alpha}$
\EndProcedure
\end{algorithmic}
\end{algorithm}


Algorithm \ref{alg:numsub} is the algorithm that separate the whole interval containing all singular values into equal-number subinterval.
\begin{algorithm}
%\small
\caption{Equal Number of Singular Value Subinterval Algorithm}
\label{alg:numsub}
\begin{algorithmic}[1]
\Procedure{$\mathbf{Number\_Divide}$}{$n, B, t, \tau$}
  \State Obtain singular value boundary $[l,u)$ of matrix $B$;
  \State Define the thread ID $i$, $i<t$;
  \Comment t is total threads
  \State $mid = inside(l, u, B)$;
  \State $n_m = NegCount(n, B, mid)$;
  \While {$n_m \ne (i+1)n/t$ and $mid-l > \tau$}
    \If {$n_m \ge (i+1)n/t$}
      \State $u=mid$;
      \State $n_u=n_m$;
    \Else
      \State $l=mid$;
      \State $n_l=n_m$;
    \EndIf
    \State $mid = inside(l, u, B)$;
    \State $n_m = NegCount(n, B, mid)$;
  \EndWhile
  \State save the division point $mid$ and $n_m$.
\EndProcedure
\end{algorithmic}
\end{algorithm}


\vspace{-0.1in}
\bibliographystyle{splncs}
\bibliography{ref}

\end{document}
