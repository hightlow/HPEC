\documentclass{llncs}

\usepackage{ifpdf}
\ifpdf
  \usepackage[pdftex]{graphicx}
  \graphicspath{{./pdf/}{./jpeg/}{./eps/}}
  \DeclareGraphicsExtensions{.pdf,.jpeg,.png}
\else
  \usepackage[dvips]{graphicx}
  \graphicspath{{./eps/}}
  \DeclareGraphicsExtensions{.eps}
\fi

\usepackage{color}
\usepackage[cmex10]{amsmath}
\usepackage{amsfonts}
\usepackage[noend]{algpseudocode}
\usepackage{algorithm}
\algnewcommand{\LineComment}[1]{\State \(\triangleright\) #1}
\usepackage[tight,footnotesize]{subfigure}
\begin{document}

\title{Attachment: Bisection and Twisted Bidiagonal SVD on GPU for Big Matrices}
\section{Algorithms}
Algorithm \ref{alg:negcount} is the $NegCount()$ function. It calculates the singular value that are less than $\mu$ of a $n\times n$ matrix $B$.
\input{algorithm_negcount.tex}

Algorithm \ref{alg:bisection} is the Bisection Algorithm. It gets the singular values in the interval of $[l,u)$ of a $n\times n$ matrix $B$. $val$ is the output of the function containing the singular value between $[l,u)$. $n_l$ and $n_u$ are the indeces of singular values.
\input{algorithm_bisection.tex}

\vspace{1in}
Algorithm \ref{alg:twisted} is the algorithm calculates the singular vectors.
\input{algorithm_twisted.tex}
\begin{equation}
\label{eq:gamma}
k = \arg \min_{1\le i \le n} \gamma_{i} = \arg \min_{1\le i \le n}
\begin{cases}
\beta_1 & \text{if } i=1 \\
\beta_i - \alpha_i * l_{i-1} & \text{if } 2\le i\le n-1\\
\alpha_n & \text{if } i=n
\end{cases}
\end{equation}
where $\alpha_i$, $\beta_i$, and $l_i$ are mentioned in Section Algorithm.

\vspace{1in}
Algorithm \ref{alg:lengthsub} is the algorithm that separate the whole interval containing all singular values into equal-length subinterval.
\input{algorithm_lengthsub.tex}

Algorithm \ref{alg:numsub} is the algorithm that separate the whole interval containing all singular values into equal-number subinterval.
\input{algorithm_numsub.tex}

\vspace{-0.1in}
\bibliographystyle{splncs}
\bibliography{ref}

\end{document}
