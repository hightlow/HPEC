\section{Implementation and Optimization on GPU} \label{sec:implementation}
\subsection{Singular Value Kernels} \label{sec_svalue}
To obtain the singular values, we divide the whole interval containing all the singular values into several subintervals, which can be computed in parallel. 
The whole interval can be calculated with Gershgorin circle theorem \cite{gershgorin}.
Each subinterval can then be assigned to one GPU block (a group of threads executed concurrently on a CUDA multiprocessor unit) for calculation.
Because there is a limit on the maximum number of threads for each GPU block,
the number of singular values in a subinterval must not exceed that maximum number. Otherwise, the computation will have to be wrapped into two or more sequential phases to complete.
Therefore how to divide the interval is critical for the performance of
computing singular values. We consider two different partition
strategies as described below. 
\begin{figure}[hbpt]
\vspace{-0.15in}
\centering
  \subfigure[Equal Length Division]
  {
  \includegraphics[width=0.4\textwidth]{length_interval}
  \label{fig:length_interval}
  }
  \\
  \vspace{-0.15in}
  \subfigure[Equal Number Division]
  {
  \includegraphics[width=0.4\textwidth]{number_interval}
  \label{fig:number_interval}
  }
\vspace{-0.1in}
\caption{Division Strategies for Partitioning Interval  $[a_0, a_n)$.} 
\vspace{-0.1in}
\end{figure}

The first (na\"{\i}ve) strategy is to divide the interval into chunks of equal length as shown in Fig.\ref{fig:length_interval}.
The whole interval $[a_0, a_n)$ is divided into $n$ subintervals with
the same length (i.e. same number of matrix elements). However, each
such subinterval might not contain the same number of singular values.
Mathematically, this strategy can be expressed as
$a_{k+1}-a_k = a_{k}-a_{k-1}$. However, $NegCount(b_{k+1})-NegCount(b_{k}) = NegCount(b_{k})-NegCount(b_{k-1})$ is not necessarily true. We call this strategy as {\it equal length division}.

Such ``equal length division'' strategy has issues in load balancing on 
parallel computing resources, as the number of singular values in each
subinterval is not uniform. As a result,
the SVD workload cannot be easily balanced on GPU's processing cores to achieve a satisfactory speedup because the optimal number of subintervals is dependent on multiple factors, such as the maximum number of threads, the number of threads in a GPU warp, and the size of matrix.
The maximum number of threads per block determines the minimum number
of subintervals that are assigned to corresponding GPU blocks.
In addition, the number of threads in a GPU warp assigned to a subinterval affects the GPU utilization level.

We have conducted experiments to understand the trade-offs between subinterval length and the GPU execution efficiency. 
Smaller subintervals will lead to a larger number of subintervals, which cause the GPU warp execution to be less efficient. 
For example, our experiments show that if $512$ subintervals are used, the efficiency of GPU warp execution ranges from only 12.38\% to 52.15\% when the matrix size increases from $1,000$ to $17,000$.
That is, only half of the threads in a warp work in parallel at most. This is because  (1)
 there are fewer singular values than the number of warps, leading the cores under-utilized; and 
(2) the warps are stalled due to memory accesses.
When the subinterval becomes larger, some block may get more singular values to calculate than its capability.
In order to achieve a higher speedup, we could try to find the optimal number of subintervals, or the number of blocks, based on experimental results. 
However, because of the uneven number of singular values within subintervals, the speedup for equal length division remains very limited, which motivates us to search for a better partition method.

Our novel strategy is to divide the whole interval by the number of singular values, as shown in Fig.\ref{fig:number_interval}. We call this approach {\it equal number division}. 
Specifically, we divide the interval $[b_0,b_n)$ into $n$ subintervals, each of which has the same number of singular values. 
However, the length of a subinterval is not necessarily equal to others.
The mathematical representation can be written as $NegCount(a_{k+1})-NegCount(a_{k})=NegCount(a_{k})-NegCount(a_{k-1})$ but $b_{k+1}-b_k = b_{k}-b_{k-1}$ does not necessarily hold anymore.
This equal number division significantly improves the load balancing on GPU cores, helping achieve a better speedup, as we show the performace evaluation in Figure \ref{fig:compare_value_kernel}.

The significant advantage for the equal number division strategy is that, more singular values can be calculated with the equal number division strategy than equal length division strategy.
Our experiment results show that equal number division outperforms length division strategy:
The warp execution efficiency in equal number division version can reach 92.97\%-96.13\%.

\subsection{Singular Vector Kernel} \label{sec_svector}
Since the singular vectors are two $n\times n$ dense matrices, as the matrix size increases, there is increasing pressure on memory storage because the size of the fast shared memory is limited in GPU architectures.
It is impracticable to move all the matrix entries to the shared memory when matrix size becomes extremely large.

Based on the twisted algorithm, the required memory size $M$ on GPU is determined by matrix size $n$ and the size of floating number $S$ in equation $M = 5 * S * n^2$, 5 is the number of arrays required for a pair of singular vectors.
For example, the maximum matrix size with $4$-bytes floating number values is about $10000$ for a GPU with $2$GB GPU memory. 
For matrices larger than that, we need to explore multiple GPUs to scale its performance as described in Section \ref{sec_mgpu}.

We initially implement the twisted algorithm with row-major matrix stored in the global memory
of a GPU system. 
We observe the heavy usage of global memory, which is
the slowest memory component on GPUs. 
Further analysis reveals that about 50\% of global memory transfers are read-after-write. Since it is a waste of time to read from global memory after writing to global memory,
we could improve the memory access performance by copying these values into the local memory and shared memory of the GPU.
The way to store matrix elements (column major or row major) is also critical as the memory accesses could be coalesced if the matrix elements
can be read and computed in large chunks. We present the results
of such optimizations in Section \ref{sec:results}.

\subsection{Solution to Big Matrices} \label{sec_huge}
In our design, the maximum number of singular values can be processed on a singular GPU is $m_b = subinterval\_size \times thread\_block\_size$,
while the maximum number of singular vectors are determined by $m_t = \sqrt{U / (5 * S)}$,
where $U$ is the memory size of GPU, 5 is mentioned above.
Usually, $m_b$ is much larger than $m_t$.
For Tesla K40c with 12GB memory, $m_t = 24K$, while $m_b = 262K$.
However, when the matrix size is larger than $m_t$, even larger than $m_b$,
the GPU kernels cannot obtain all singular values and vectors any more with a single GPU.
Therefore we derive a divide-and-conquer architecture to solve the huge matrix size as explained in this section.

When the size of a matrix is less than $m_b$ but larger than $m_t$,
we only need to divide the singular vectors into small sets, each of which can be processed by a single GPU.
However, if the matrix size is larger than $m_b$, we should divide both singular values and singular vectors into smaller partitions.
The singular value computations can be partitioned using $m_b$ directly, i.e., there are $l_b = \lfloor(n/m_b)\rfloor + 1$ partitions, each of which has $\lfloor(n/l_b)\rfloor$ singular values.

The division of singular vectors should take memory size into consideration.
The maximum number of partitions $l_t$ can be derived from Equation $l_t = \sqrt{U/(5 * n * 4)}$, 
where $n$ is matrix size.
Thus, there should be $\lfloor(n/l_t)\rfloor+1$ partitions with
$\lfloor(n/l_t)\rfloor$ singular vectors in each partition.

\subsection{Multiple GPUs on a server}\label{sec_mgpu}
We implement the multiple-GPU version with Pthread libraries on one server
to control data partition and assignment to GPUs.
Pthread is a nature choice for controlling multiple GPUs on a single server
because it uses shared memory architecture, dramatically reducing overheads
on data sharing. We also design an extensible interface between CPU and GPU,
 which is easy to scale to physical distributed GPUs, compared to CUDA asynchronization interface. In our design, each thread takes control of one GPU.

When matrix size becomes huge, the load balancing among multiple GPUs will be a problem for performance as shown in column 2 of Table \ref{tab:hugeResultTesla}.
To conquer this issue we also introduce a dynamic load-balancing method for a better speedup, where a GPU will take new tasks on unfinished subintervals as soon as it finishes its prior task.

